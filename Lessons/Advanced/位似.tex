\part{位似}
\section{位似变换}

\begin{definition}[位似变换]
    一个位似 $h$ 是一个变换,依赖一个位似中心 $O$ 和一个常数 $k$. 此变换将一个点 $P$ 映射到点 $h(P)$,点 $P$ 到 $O$ 的距离被乘以了 $k$. 称 $k$ 为这个位似变换的缩放因子(或位似系数)。

    
\end{definition}

\begin{proposition}
    缩放因子大于零的位似变换称为外位似变换(或正向位似); 缩放因子小于零的位似变换称为内位似变换(或负向相似)。相应的,位似中心称为外位似中心和内位似中心.  
\end{proposition}

\begin{proposition}
    位似变换 $h$ 将 $A$ 映射为 $A'$,则一定有:

    (1) $O, A, A'$ 三点共线。

    (2) $\overrightarrow{OA'} = k \cdot \overrightarrow{OA},$ 其中 $k$ 为缩放因子。
\end{proposition}


\begin{lemma}[位似三角形]
    设 $\triangle ABC$ 和 $\triangle XYZ$ 不全等,满足 $AB \parallel XY$,$BC \parallel YZ$,$CA \parallel ZX$。证明:直线 $AX, BY, CZ$ 交于一点 $O$,并且 $\triangle ABC$ 和 $\triangle XYZ$ 以 $O$ 为中心位似。
\end{lemma}


\begin{exercise}
    设$\triangle ABC$的内切圆 $I$ 和A-旁切圆 $J$ 分别与 $BC$ 边相切于 $D, E$. $K$ 为 $D$ 在圆$I$上的对径点,$M$ 为 $BC$ 边上中点。证明:$A,K,E$三点共线,$IM \parallel KE.$
\end{exercise}