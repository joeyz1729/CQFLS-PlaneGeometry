\part{圆与三角形五心}

\begin{exercise}
    习题 2.18. 设 $\triangle ABC$ 在顶点 $B, C$ 处的外角平分线相交于 $I_A$。证明 $I_A$ 是和 $\overline{BC}$,$AB$ 超过 $B$ 的延长部分,$AC$ 超过 $C$ 的延长部分都相切的某个圆的圆心,进一步,证明 $I_A$ 在射线 $AI$ 上。
\end{exercise}

\begin{exercise}
    习题 2.19. (旁切圆半径) 证明:$A$-旁切圆和 $A$-旁切圆的半径为 $r_a = \frac{s}{s-a}r$。提示:302.
\end{exercise}


\begin{exercise}
    习题 2.20. 设 $\triangle ABC$ 的内切圆和 $A$-旁切圆在 $BC$ 上的切点分别是 $D, X$。证明:$BX = CD, BD = CX$。
\end{exercise}


\begin{exercise}
    例 2.21. (USAMO 2009/1) 两圆 $\omega_1, \omega_2$ 的圆心与 $\omega_1$ 相交于点 $X, Y$。直线 $l_1$ 经过 $\omega_1$ 的圆心与 $\omega_2$ 相交于 $P, Q$;直线 $l_2$ 经过 $\omega_2$ 的圆心与 $\omega_1$ 相交于 $R, S$。证明:若 $P, Q, R, S$ 共圆,则此圆圆心在直线 $XY$ 上。
\end{exercise}


\begin{exercise}
    引理 2.22. (关于内外径的欧拉定理) 设 $\triangle ABC$ 的外接圆半径和内切圆半径分别为 $R, r$。若 $O, I$ 分别是二者的圆心,则 $OI^2 = R(R - 2r)$。特别地,$R \geq 2r$。
\end{exercise}


\begin{exercise}
引理 2.24. 设 $\triangle ABC$ 的内心为 $I$,旁切圆圆心为 $I_A, I_B, I_C$。证明:$\triangle I_A I_B I_C$ 的垂心为 $I$,而 $\triangle ABC$ 是它的垂足三角形。提示:564, 103.
\end{exercise}


\begin{exercise}
    定理 2.25. (Pitot 定理) 如图 2.8A,设四边形 $ABCD$ 有一个内切圆 $I$,证明:$AB + CD = BC + DA$。提示:467.
\end{exercise}


\begin{exercise}
    习题 2.26. (USAMO 1990/5) 平面上给定锐角 $\triangle ABC$。以 $\overline{AB}$ 为直径的圆与高 $\overline{CC'}$ 及其延长线分别交于 $M, N$,以 $\overline{AC}$ 为直径的圆与高 $\overline{BB'}$ 及其延长线分别交于 $P, Q$。证明:$M, N, P, Q$ 四点共圆。提示:260, 73, 409. 答案:第248 页。
\end{exercise}

\begin{exercise}
    习题 2.27. (BAMO 2012/4) 给定平面上的线段 $\overline{AB}$,在在线段上选择不同于 $A, B$ 的一点 $M$。平面上两个等边 $\triangle AMC$ 和 $\triangle BMD$ 在线段 $\overline{AB}$ 的同一侧,两个三角形的外接圆交于点 $M$ 和另外一点 $N$。
(a) 证明:$\overline{AD}$ 和 $\overline{BC}$ 经过点 $N$。提示:57, 77.
(b) 证明:当 $M$ 在线段 $\overline{AB}$ 上移动时,所有的直线 $MN$ 经过平面上某固定点 $K$。提示:230, 654.
\end{exercise}

\begin{exercise}
    习题 2.28. (JMO 2012/1) 给定 $\triangle ABC$,设 $P, Q$ 分别是线段 $\overline{AB}, \overline{AC}$ 上的点,满足 $AP = AQ$。设 $S, R$ 是线段 $\overline{BC}$ 上的不同点,$S$ 在 $B, R$ 之间,$\angle BPS = \angle PRS$,$\angle CQR = \angle QSR$。证明:$P, Q, R, S$ 四点共圆。提示:435, 601, 537, 122.
\end{exercise}

\begin{exercise}
    习题 2.29. (IMO 2008/1) 设 $H$ 是锐角 $\triangle ABC$ 的垂心。圆 $\Gamma_A$ 以 $\overline{BC}$ 的中点为圆心,过点 $H$,交直线 $\overline{BC}$ 于点 $A_1, A_2$。类似地定义点 $B_1, B_2, C_1, C_2$。证明:六个点 $A_1, A_2, B_2, B_2, C_1, C_2$ 共圆。提示:82, 597. 答案第248 页。
\end{exercise}

\begin{exercise}
    习题 2.30. (USAMO 1997/2) 给定 $\triangle ABC$,点 $D, E, F$ 分别在边 $\overline{BC}, \overline{CA}, \overline{AB}$ 的垂直平分线上。证明:过 $A, B, C$ 分别垂直于 $\overline{EF}, \overline{FD}, \overline{DE}$ 的直线共点。提示:100, 3, 611.
\end{exercise}

\begin{exercise}
    习题 2.31. (IMO 1995/1) 设 $A, B, C, D$ 是一条直线上的依次四点。以 $\overline{AC}$ 和 $\overline{BD}$ 为直径的圆相交于 $X, Y$,直线 $XY$ 交 $\overline{BC}$ 于 $Z$,点 $P$ 是 $XY$ 上不同于 $Z$ 的一点,直线 $CP$ 与以 $\overline{AC}$ 为直径的圆交于 $C, M$,直线 $BP$ 与以 $\overline{BD}$ 为直径的圆交于 $B, N$。证明:直线 $AM, DN, XY$ 三线共点。提示:49, 159, 134.
\end{exercise}

\begin{exercise}
    习题 2.32. (USAMO 1998/2) 已知 $C_1$ 和 $C_2$ 是两个同心圆 ($C_2$ 在 $C_1$ 内)。点 $A$ 在 $C_1$ 上,$AB$ 为 $C_1$ 上任意一点,过 $A$ 引 $C_2$ 的切线 $AB$ ($B \in C_2$),交 $C_1$ 于一点 $C$,取 $AB$ 为 $C_1$ 上任意一点,过 $A$ 引 $C_2$ 的切线 $AB$ 于点 $E$ 和 $F$,使得 $DE$ 和 $CF$ 的中垂线交于中点 $D$。过 $A$ 引一条直线交 $C_2$ 于点 $E$ 和 $F$,使得 $DE$ 和 $CF$ 的中垂线交于 $D$。求 $\frac{AM}{MC}$ 的值,并予以证明。提示:659, 355, 482。
\end{exercise}

\begin{exercise}
    习题 2.333. (IMO 2000/1) 圆 $\Gamma_1$ 和圆 $\Gamma_2$ 相交于点 $M$ 和 $N$。设直线 $AB$ 与 $G_2$ 分别相切于 $A, B$,并且 $M$ 距离 $AB$ 比 $N$ 近。设直线 $CD$ 经过点 $M$ 且与 $G_2$ 分别相切于 $A, B$,并且 $M$ 距离 $AB$ 比 $N$ 近。设直线 $AN$ 和 $AB$ 平行,$C$ 在 $G_1$ 上,$D$ 在 $G_2$ 上。直线 $CA$ 和 $DB$ 相交于点 $E$,直线 $AN$ 和 $CD$ 相交于点 $P$,直线 $BN$ 和 $CD$ 相交于点 $Q$。证明:$EP = EQ$。提示:17, 174.
\end{exercise}

\begin{exercise}
    习题 2.34. (加拿大 1990/3) 设圆内接四边形 $ABCD$ 的对角线相交于 $P$。设 $W, X, Y, Z$ 分别是 $P$ 到 $\overline{AB}, \overline{BC}, \overline{CD}, \overline{DA}$ 的投影。证明:$WX + YZ = XY + WZ$。提示:1, 414, 440。答案:第249 页。
\end{exercise}

\begin{exercise}
    习题 2.35. (IMO 2009/2) 设 $\triangle ABC$ 的外接圆圆心为 $O$。点 $P, Q$ 分别是线段 $\overline{CA}, \overline{AB}$ 内的点,点 $K, L, M$ 分别是线段 $BP, CQ, PQ$ 的中点,圆 $F$ 经过 $K, L, M$。假设直线 $PQ$ 与 $F$ 相切,证明:$OP = OQ$。提示:78, 544, 346。
\end{exercise}

\begin{exercise}
习题 2.36. 设 $AD, BE, CF$ 是不等边 $\triangle ABC$ 的三条高,$O$ 是 $\triangle ABC$ 的外心。证明:三个圆 $(AOD), (BOE), (COF)$ 相交于不同于 $O$ 的另外一点 $X$。提示:553, 79。答案:第250 页。

\end{exercise}


\begin{exercise}
习题 2.37. (加拿大 2007/5) 设 $\triangle ABC$ 的内切圆与边 $BC, CA, AB$ 分别相切于 $D, E, F$,设 $\omega_1, \omega_2, \omega_3$ 分别是 $\triangle ABC, \triangle AEF, \triangle BDF, \triangle CDE$ 的外接圆。设 $\omega$ 和 $\omega_1$ 交于 $A, P$;$\omega$ 和 $\omega_2$ 交于 $B, Q$;$\omega$ 和 $\omega_3$ 交于 $C, R$。
(a) 证明:$\omega_1, \omega_2, \omega_3$ 交于一点。
(b) 证明:直线 $PD, QE, RF$ 三线共点。提示:376, 548, 660。
    
\end{exercise}


\begin{exercise}
习题 2.38. (伊朗 TST 2011/1) 在锐角 $\triangle ABC$ 中,$\angle B > \angle C$。设 $M$ 是 $\overline{BC}$ 的中点,$E, F$ 分别是从 $B, C$ 出发的高的垂足。设 $K, L$ 分别是 $ME, MF$ 的中点。点 $T$ 在直线 $KL$ 上,满足 $TA \parallel BC$。证明:$TA = TM$。提示:297, 495, 154。答案:第250 页。
    
\end{exercise}
